\documentclass[10pt,a4paper]{report}
\usepackage[utf8]{inputenc}
\usepackage{amsmath}
\usepackage{amsfonts}
\usepackage{amssymb}
\usepackage{graphicx}
\usepackage{pgfplots}
\usepackage{pgfplotstable}

\usepackage{graphicx}
\usepackage{epstopdf}

\usetikzlibrary{pgfplots.groupplots}

%\usepackage{struktex}
\usepackage{hyperref}

\usepackage[left=1.5cm,right=1.5cm,top=2cm,bottom=2cm]{geometry}



\definecolor{s1_1}{RGB}{38,70,83}
\definecolor{s1_2}{RGB}{42,157,143}
\definecolor{s1_3}{RGB}{233,196,106}
\definecolor{s1_4}{RGB}{244,162,97}
\definecolor{s1_5}{RGB}{231,111,81}

\definecolor{s2_1}{RGB}{246,81,29}
\definecolor{s2_2}{RGB}{255,180,0}
\definecolor{s2_3}{RGB}{0,166,237}
\definecolor{s2_4}{RGB}{127,184,0}
\definecolor{s2_5}{RGB}{13,44,84}



\pgfplotscreateplotcyclelist{convergelist}{
s2_1, thick, solid, mark=*\\%
s2_2, thick, solid, mark=square\\%
s2_3, thick, solid, mark=diamond*\\%
s2_4, thick, solid, mark=triangle\\%
s2_5, thick, solid, mark=asterisk\\%
}

\pgfplotscreateplotcyclelist{elelist}{
s2_1, solid\\%
s2_2, solid\\%
s2_3, solid\\%
s2_4, solid\\%
s2_5, solid\\%
}


%\pgfplotsutilstrreplace{_}{\_aa}{##1}

%\usepackage{syntax}[nounderscore]


\def \resultspath {../results}

\def \foobar {config_task3_K5_N1_LF}


\newcommand{\dvec}[1]{\boldsymbol{ \mathsf{#1} } }         % for vectors
\newcommand{\dmat}[1]{\boldsymbol{\mathsf{#1}}}           % for matrices

\newcommand{\Bd}[2]{ (#1,#2)_{D^k} }           % bilienarfor over domian
\newcommand{\Bb}[2]{ (#1,#2)_{\partial D^k} }           % bilienarfor over boundary


%\newcommand{\resultspath}{home/lukas/LRZ Sync+Share/Studium/Fach/DG/DG\_project/lukas/results}
%\newcommand{\resultspath}{home/lukas/LRZ Sync+Share/Studium/Fach/DG/DG_project/lukas/results}
\newcommand{\getResultFile}[1]{\resultspath/#1}
\newcommand{\getEleFile}[2]{\getResultFile{#1}\_ele#2.dat}

%home/lukas/LRZ Sync+Share/Studium/Fach/DG/DG_project/lukas


\title{Report Project 2}
\author{Lukas Scheucher}



\pgfplotstableset{
  col sep=comma,
    create on use/X/.style={create col/copy column from table={/home/lukas/LRZ Sync+Share/Studium/Fach/DG/DG_project/lukas/results/config_task3_K5_N1_LF_x.dat}{0}}
}


%\pgfplotstableread{/home/lukas/LRZ Sync+Share/Studium/Fach/DG/DG_project/lukas/results/config_task3_K5_N1_LF_V_ele2.dat}\Vtable
%\pgfplotstableread{/home/lukas/LRZ Sync+Share/Studium/Fach/DG/DG_project/lukas/results/config_task3_K5_N1_LF_U.dat}\Utable
%\pgfplotstableread{/home/lukas/LRZ Sync+Share/Studium/Fach/DG/DG_project/lukas/results/config_task3_K5_N1_LF_t.dat}\ttable
%\pgfplotstableread{/home/lukas/LRZ Sync+Share/Studium/Fach/DG/DG_project/lukas/results/config_task3_K5_N1_LF_x.dat}\xtable



\begin{document}
\maketitle
aaa
\part{aSDAs}

\chapter*{Reuslts}
\section*{Task3}
a
\section*{Task4}


\begin{figure}
\begin{center}
\begin{tikzpicture}

\begin{loglogaxis}[xlabel={$numele$},ylabel={$\epsilon_p$}, grid=major,width=12cm,height=7cm, legend pos=outer north east,cycle list name=convergelist]

      \addplot+ table[x index=0,y index=1] {\resultspath/config_task3_convergence_N1_LF.dat};
      \addplot+ table[x index=0,y index=1] {\resultspath/config_task3_convergence_N2_LF.dat};
      \addplot+ table[x index=0,y index=1] {\resultspath/config_task3_convergence_N3_LF.dat};
      \addplot+ table[x index=0,y index=1] {\resultspath/config_task3_convergence_N4_LF.dat};
      \pgfplotsset{cycle list shift=-4}
      \addplot+[dashed] table[x index=0,y index=1] {\resultspath/config_task3_convergence_N1_HDG.dat};
      \addplot+[dashed] table[x index=0,y index=1] {\resultspath/config_task3_convergence_N2_HDG.dat};
      \addplot+[dashed] table[x index=0,y index=1] {\resultspath/config_task3_convergence_N3_HDG.dat};
      \addplot+[dashed] table[x index=0,y index=1] {\resultspath/config_task3_convergence_N4_HDG.dat};
      \legend{N=1 (LF),N=2 (LF),N=3 (LF),N=4 (LF),N=1 (HDG),N=2 (HDG),N=3 (HDG),N=4 (HDG)}
\end{loglogaxis}

\end{tikzpicture}
\end{center}
\caption{Comparison of LF and HDG flux for task 3. From this setup one can not really determine a superior flux.}
\label{fig:task3_flux_comparison}
\end{figure}





\begin{figure}[h]
\begin{center}
\includegraphics[width=0.9\textwidth]{\resultspath/config_task3_K200_N1_hdg.pdf}
\label{fig:task3_imagesc_hdg}
\caption{Setup as described in \ref{fig:setup3} with dirichlet boundary conditions. The x-axis shows the position, the y-axis the time. A perfect periodic oscillation can be observed. For this plot 200 elements with a polymomial degree of 1 were used.}
\end{center}
\end{figure}



\end{document}